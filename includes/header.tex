%%% begin header.tex
% \makeatletter
% \providecommand*{\input@path}{}
% % For latex-workshop only
% \edef\input@path{{../convert/pandoc/includes/}{../pandoc/}{./../}\input@path}%
% % Normal paths
% \edef\input@path{{convert/pandoc/includes/}{pandoc/}{./}\input@path}% prepend
% \makeatother

% Language ==========================================
\usepackage{microtype}
%\usepackage[american]{babel}
\selectlanguage{american}
\input{hyphenex/ushyphex.tex}
\usepackage{fontspec}

% Calculations =======================================
\usepackage{calc}

% Some booleans ======================================
\usepackage{ifthen}
\newboolean{printVersion}
\setboolean{printVersion}{true}


%% Numerals ==========================================
\usepackage{nth}

%% Geometry ==========================================
\newlength{\innermargin}
\setlength{\innermargin}{25mm}
\usepackage[nomarginpar,
            textwidth=155mm,
            textheight=240mm,
            bottom=30mm,
            headheight=15pt,
            headsep=20pt,
            footskip=30pt,
            bindingoffset=5mm,
            inner=\innermargin
            ]{geometry}
\setlength{\columnsep}{0.95cm}
\usepackage[final]{pdfpages}

%% Grafics ==========================================
\usepackage{graphicx}
\usepackage[export]{adjustbox}
\usepackage{tikz}
\usepackage{wrapfig}
\usepackage[absolute]{textpos}

%% Caption ===========================================
\usepackage[%
    format=plain,
    justification=centering,
    margin=12pt,
    font=normal,
    labelfont=bf
]{caption}

\DeclareCaptionStyle{figurecaption}{
    format=plain,
    justification=centering,
    margin=12pt,
    font=normal,
    labelfont=bf,
    position=bottom,
    aboveskip=10pt,
    belowskip=-5pt
}

\DeclareCaptionStyle{tablecaption}{
    format=plain,
    justification=centering,
    margin=12pt,
    font=normal,
    labelfont=bf,
    position=bottom,
    aboveskip=5pt,
    belowskip=5pt
}

\usepackage[]{subcaption}
\captionsetup[figure]{style=figurecaption}
\captionsetup[table]{style=tablecaption}

% continuous figure and table numbering
\counterwithout{figure}{chapter}
\counterwithout{table}{chapter}

% set caption label style to arabic
\renewcommand{\thefigure}{\arabic{figure}}

% Footnotes
\usepackage[perpage,para]{footmisc}
% ====================================================

%% Tables ============================================
\usepackage{array}
%\usepackage{ctable}
\usepackage{multirow}
\usepackage{tabularx}
\usepackage{booktabs}
\usepackage{longtable}
\usepackage{supertabular}
\usepackage{ltxtable}
\usepackage{collcell}
\usepackage{makecell}
\usepackage{colortbl}
\usepackage[svgnames]{xcolor}
% ====================================================

%% Use Links & Refs ==================================
\ifthenelse{\boolean{printVersion}}{
    \colorlet{myurlcolor}{gray!50!black}%
    \colorlet{mylinkcolor}{gray!50!black}%
    \colorlet{mycitecolor}{gray!50!black}%
    \colorlet{myanchorcolor}{gray!50!black}%
    \colorlet{mybackrefcolor}{gray!10!black}%
}{%
    \colorlet{myurlcolor}{blue!50!black}%
    \colorlet{mylinkcolor}{blue!50!black}%
    \colorlet{mycitecolor}{blue!50!black}%
    \colorlet{myanchorcolor}{blue!50!black}%
    \colorlet{mybackrefcolor}{green!35!black}%
}

% Loaded from pandoc
% \usepackage[hidelinks,
% colorlinks = true,
% linkcolor = black,
% urlcolor  = myurlcolor,
% citecolor = black,
% anchorcolor = black]{hyperref}

%% Code Blocks ========================================
\usepackage{listings}
\lstset{numberbychapter=false}

% customize highlighting environments (set fancyvrb options)
%\RecustomVerbatimEnvironment{Highlighting}{Verbatim}{
%    frame=single,
%    fontsize=\footnotesize,
%    commandchars=\\\{\},
%}

\lstdefinelanguage{Go}{
    % Keywords as defined in the language grammar
    morekeywords=[1]{%
        break,default,func,interface,select,case,defer,go,map,%
        struct,chan,else,goto,package,switch,const,fallthrough,%
        if,range,type, continue,for,import,return,var},
    % Built-in functions
    morekeywords=[2]{%
        append,cap,close,complex,copy,delete,imag,%
        len,make,new,panic,print,println,real,recover},
    % Pre-declared types
    morekeywords=[3]{%
        bool,byte,complex64,complex128,error,float32,float64,%
        int,int8,int16,int32,int64,rune,string,%
        uint,uint8,uint16,uint32,uint64,uintptr},
    % Constants and zero value
    morekeywords=[4]{true,false,iota,nil},
    % Strings : "foo", 'bar', `baz`
    morestring=[b]{"},
    morestring=[b]{'},
    morestring=[b]{`},
    % Comments : /* comment */ and // comment
    comment=[l]{//},
    morecomment=[s]{/*}{*/},
    % Options
    sensitive=true
}

\lstdefinelanguage{yaml}{
%    % Keywords as defined in the language grammar
%    morekeywords=[1]{%
%        break,default,func,interface,select,case,defer,go,map,%
%        struct,chan,else,goto,package,switch,const,fallthrough,%
%        if,range,type, continue,for,import,return,var},
%    % Built-in functions
%    morekeywords=[2]{%
%        append,cap,close,complex,copy,delete,imag,%
%        len,make,new,panic,print,println,real,recover},
%    % Pre-declared types
%    morekeywords=[3]{%
%        bool,byte,complex64,complex128,error,float32,float64,%
%        int,int8,int16,int32,int64,rune,string,%
%        uint,uint8,uint16,uint32,uint64,uintptr},
    % Constants and zero value
    morekeywords=[4]{true,false,null,y,n},
    morestring=[b]{"},
    morestring=[b]{'},
    comment=[l]{\#},
    morecomment=[s]{/*}{*/},
    sensitive=false
}

%\newcommand\YAMLcolonstyle{\color{red}}
%\newcommand\YAMLkeystyle{\color{black}}
%\newcommand\YAMLvaluestyle{\color{blue}}
%
%\makeatletter
%
%% here is a macro expanding to the name of the language
%% (handy if you decide to change it further down the road)
%\newcommand\language@yaml{yaml}
%
%\expandafter\expandafter\expandafter\lstdefinelanguage
%\expandafter{\language@yaml}
%{
%    keywords={true,false,null,y,n},
%    keywordstyle=\color{darkgray},
%    basicstyle=\ttfamily\footnotesize,                                 % assuming a key comes first
%    sensitive=false,
%    comment=[l]{\#},
%    morecomment=[s]{/*}{*/},
%    commentstyle=\color{purple},
%    stringstyle=\YAMLvaluestyle,
%    moredelim=[l][\color{orange}]{\&},
%    moredelim=[l][\color{magenta}]{*},
%    moredelim=**[il][\YAMLcolonstyle{:}\YAMLvaluestyle]{:},   % switch to value style at :
%    morestring=[b]',
%    morestring=[b]",
%    literate =    {---}{{\textcolor{black}}}3
%        {>}{{\textcolor{red}\textgreater}}1
%        {|}{{\textcolor{red}\textbar}}1
%        {\ -\ }{{\ -\ }}3,
%}
%
%% switch to key style at EOL
%\lst@AddToHook{EveryLine}{\ifx\lst@language\language@yaml\YAMLkeystyle\fi}
%\makeatother


\setmonofont{JetBrainsMono}[
    Path=./pandoc/fonts/,
    Scale=0.85,
    Extension = .ttf,
    UprightFont=*-Regular,
    BoldFont=*-Bold,
    ItalicFont=*-Italic,
    BoldItalicFont=*-BoldItalic
]

\definecolor{codegreen}{rgb}{0,0.6,0}
\definecolor{codegray}{rgb}{0.5,0.5,0.5}
\definecolor{codepurple}{rgb}{0.58,0,0.82}
\definecolor{codebackground}{rgb}{0.95,0.95,0.92}

\lstdefinestyle{mycodestyle}{
    backgroundcolor=\color{WhiteSmoke},
    commentstyle=\color{codegreen},
    keywordstyle=\color{magenta},
    numberstyle=\color{codegray},
    stringstyle=\color{codepurple},
    basicstyle=\ttfamily\footnotesize,
    breakatwhitespace=false,
    breaklines=true,
    captionpos=b,
    keepspaces=true,
    numbers=none,
    numbersep=10pt,
    showspaces=false,
    showstringspaces=false,
    showtabs=false,
    tabsize=1
}

\lstset{style=mycodestyle}

%% Quotes =============================================
\usepackage[font=itshape,vskip=5pt]{quoting}

% Hypenate tt text
\usepackage[htt]{hyphenat}

%% Title & Headings ==================================

\renewcommand*\chapterheadstartvskip{\vspace*{0pt}}
\renewcommand*\chapterheadendvskip{\vspace*{12pt}}

% Tocless Command ====================================
\newcommand{\nocontentsline}[3]{}
\newcommand{\tocless}[2]{\bgroup\let\addcontentsline=\nocontentsline#1{#2}\egroup}

% Make some floats adjustments =======================
% make floatbarrier at each section
\usepackage[section]{placeins}

% spacing between floats and text/floats
% \setlength{\textfloatsep}{20pt plus 2pt minus 4pt}
% \setlength{\floatsep}{12pt plus 2pt minus 2pt}
\setlength{\intextsep}{20pt plus 3pt minus 3pt}

% \setcounter{topnumber}{2}
% \setcounter{bottomnumber}{2}
% \setcounter{totalnumber}{3}     % 2 may work better
% \setcounter{dbltopnumber}{2}    % for 2-column pages
\renewcommand{\topfraction}{0.85}
\renewcommand{\bottomfraction}{0.8}
\renewcommand{\textfraction}{0.07}
\renewcommand{\floatpagefraction}{0.8}
%\renewcommand{\dbltopfraction}{.66}
%\renewcommand{\dblfloatpagefraction}{.8}

% Allow display break
%\allowdisplaybreaks[3]
%=====================================================

% Clever Refs
\usepackage{hyperref}
\usepackage{cleveref} % load it after thmtools (there is some problem)
\crefname{appsec}{appendix}{appendices}

\KOMAoptions{cleardoublepage=empty}

%=====================================================
%custom stuff ========================================
%=====================================================

\usepackage{lipsum}
% include list of figures, etc. in TOC
\usepackage[
    nottoc, % exclude TOC ref from TOC
    ]{tocbibind}

\usepackage{todonotes}

%%% custom title page
\newcommand{\TitlePageTitle}{Towards Horizontally Scalable Kubernetes Controllers}
\newcommand{\ShortTitle}{Towards Horizontally Scalable Kubernetes Controllers}
\newcommand{\Degree}{Master of Science (M.Sc.)}
\newcommand{\Course}{Informatik}
\newcommand{\EnrollmentNumber}{8559152}
\newcommand{\Company}{SAP SE}
\newcommand{\Lecturer}{Prof. Dr.-Ing. habil. Dennis Pfisterer}
\newcommand{\AdditionalTitlePageLogo}{}
% \newcommand{\AdditionalTitlePageLogo}{\fbox{\includegraphics[width=3cm]{logo/...}}}
\newcommand{\TypeOfDocument}{Studienarbeit}
\newcommand{\Date}{27.06. - 26.12.2022}

\makeatletter
\renewcommand*{\maketitle}{%
\begin{titlepage}
\begin{center}
\vspace*{-2cm}
\AdditionalTitlePageLogo\hfill\includegraphics[width=5cm]{logo/dhbw-cas.pdf}\\[2cm]
{\Huge{\TitlePageTitle}\par}
\vspace{1.5cm}
{\Huge\scshape \TypeOfDocument}\\[1cm]
{\large für die Prüfung zum}\\[0.2cm]
{\Large \Degree}\\[1.5cm]
{\large im Studiengang}\\[0.2cm]
{\large \Course}\\[1.5cm]
{\large am Center for Advanced Studies}\\[0.2cm]
{\large der Dualen Hochschule Baden-Württemberg}\\[0.5cm]
{\large von}\\[0.5cm]
{\large\bfseries \@author}\\[1cm]
\vfill
\end{center}
\begin{tabular}{l@{\hspace{2cm}}l}
Matrikelnummer & \EnrollmentNumber \\
Bearbeitungszeitraum & \Date \\
Dualer Partner & \Company \\
Prüfer & \Lecturer \\
\end{tabular}
\end{titlepage}
}
\makeatother

%%% custom header + footer
\usepackage{fancyhdr}
\fancypagestyle{plain}{% % overwrite plain pagestyle for entire document
\fancyhf{} % clear all header and footer fields
\fancyhead[L]{Tim Ebert}
\fancyhead[R]{\ShortTitle}
\fancyfoot[R]{\thepage}
\renewcommand{\headrulewidth}{0.4pt}
\renewcommand{\footrulewidth}{0.4pt}}
\pagestyle{plain}

% continue roman numbering in appendix
% https://tex.stackexchange.com/questions/166416/continue-previous-roman-page-numbering-after-changing-to-arabic
\newcounter{savepage}

%%% custom numbering for requirements
\newcounter{reqCounter}
\newcommand{\requirement}{%
\refstepcounter{reqCounter}%
Req. \arabic{reqCounter}:
}

%%% spacing
\usepackage{setspace}
\onehalfspacing % spacing between lines

\setlength{\parindent}{0pt} % paragraph indententation
\setlength{\parskip}{0.8em} % spacing between paragraphs

% % don't break page on new chapter
% \usepackage{etoolbox}
% \makeatletter
% \patchcmd{\scr@startchapter}{\if@openright\cleardoublepage\else\clearpage\fi}{}{}{}
% \makeatother

%%% end header.text
